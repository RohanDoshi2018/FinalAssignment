\documentclass{article}

\usepackage{fancyhdr}
\usepackage{extramarks}
\usepackage{amsmath}
\usepackage{amsthm}
\usepackage{amssymb}
\usepackage{amsfonts}
\usepackage{mathtools}
\usepackage{tikz}
\usepackage[plain]{algorithm}
\usepackage{algpseudocode}
\usepackage{pgfplots}
\usepackage{units}

\usetikzlibrary{automata,positioning}

%
% Basic Document Settings
%

\topmargin=-0.45in
\evensidemargin=0in
\oddsidemargin=0in
\textwidth=6.5in
\textheight=9.0in
\headsep=0.25in

\linespread{1.1}

\pagestyle{fancy}
\lhead{\hmwkAuthorName}
\chead{\hmwkClass\ \hmwkTitle}
\rhead{\firstxmark}
\lfoot{\lastxmark}
%\cfoot{\thepage}

%\chead{\hmwkClass\ (\hmwkClassInstructor\ \hmwkClassTime): \hmwkTitle}


\renewcommand\headrulewidth{0.4pt}
\renewcommand\footrulewidth{0.4pt}

\setlength\parindent{0pt}

%
% Create Problem Sections
%

\newcommand{\enterProblemHeader}[1]{
    \nobreak\extramarks{}{Problem \arabic{#1} continued on next page\ldots}\nobreak{}
    \nobreak\extramarks{Problem \arabic{#1} (continued)}{Problem \arabic{#1} continued on next page\ldots}\nobreak{}
}

\newcommand{\exitProblemHeader}[1]{
    \nobreak\extramarks{Problem \arabic{#1} (continued)}{Problem \arabic{#1} continued on next page\ldots}\nobreak{}
    \stepcounter{#1}
    \nobreak\extramarks{Problem \arabic{#1}}{}\nobreak{}
}

\newcommand*\rfrac[2]{{}^{#1}\!/_{#2}} % diagonal fraction bar

\setcounter{secnumdepth}{0}
\newcounter{partCounter}
\newcounter{homeworkProblemCounter}
\setcounter{homeworkProblemCounter}{1}
\nobreak\extramarks{Problem \arabic{homeworkProblemCounter}}{}\nobreak{}

%
% Homework Problem Environment
%
% This environment takes an optional argument. When given, it will adjust the
% problem counter. This is useful for when the problems given for your
% assignment aren't sequential. See the last 3 problems of this template for an
% example.
%
\newenvironment{homeworkProblem}[1][-1]{
    \ifnum#1>0
        \setcounter{homeworkProblemCounter}{#1}
    \fi
    \section{Problem \arabic{homeworkProblemCounter}}
    \setcounter{partCounter}{1}
    \enterProblemHeader{homeworkProblemCounter}
}{
    \exitProblemHeader{homeworkProblemCounter}
}

%
% Homework Details
%   - Title
%   - Due date
%   - Class
%   - Section/Time
%   - Instructor
%   - Author
%

\newcommand{\hmwkTitle}{Problem Set \#9}
\newcommand{\hmwkDueDate}{April 26, 2016}
\newcommand{\hmwkClass}{ECO 310}
\newcommand{\hmwkClassTime}{}
\newcommand{\hmwkClassInstructor}{Professor Morris}
\newcommand{\hmwkAuthorName}{Kevin Liu}

%
% Title Page
%

\title{
    \vspace{2in}
    \textmd{\textbf{\hmwkClass:\ \hmwkTitle}}\\
    \normalsize\vspace{0.1in}\small{Due:\ \hmwkDueDate\ }\\
    \vspace{3in}
}
%     \vspace{0.1in}\large{\textit{\hmwkClassInstructor\ \hmwkClassTime}}

\author{\textbf{\hmwkAuthorName}}
\date{}

\renewcommand{\part}[1]{\textbf{\large Part \Alph{partCounter}}\stepcounter{partCounter}\\}

%
% Various Helper Commands
%

% Useful for algorithms
\newcommand{\alg}[1]{\textsc{\bfseries \footnotesize #1}}

% For derivatives
\newcommand{\deriv}[1]{\frac{\mathrm{d}}{\mathrm{d}x} (#1)}

% For partial derivatives
\newcommand{\pderiv}[2]{\frac{\partial}{\partial #1} (#2)}

% Integral dx
\newcommand{\dx}{\mathrm{d}x}

% Alias for the Solution section header
\newcommand{\solution}{\textbf{\large Solution}}

% Probability commands: Expectation, Variance, Covariance, Bias
\newcommand{\E}{\mathbb{E}}
\newcommand{\Var}{\mathrm{Var}}
\newcommand{\Cov}{\mathrm{Cov}}
\newcommand{\Bias}{\mathrm{Bias}}

% Make floor and ceiling functions prettier
\DeclarePairedDelimiter\ceil{\lceil}{\rceil}
\DeclarePairedDelimiter\floor{\lfloor}{\rfloor}


\begin{document}

\maketitle

\pagebreak

\begin{homeworkProblem}
\begin{center}

% overview of regressions
\begin{tabular}{c c c}
	\textbf{Regression Type} & $r^2$ & \textbf{RMSE} \\
	\hline
	Ordinary Least Squares & 0.76 & 4005.25 \\
	K Nearest Neighbors & 0.42 & 6251.03 \\
	Ridge & 0.61 & 5087.40 \\
	Lasso & 0.67 & 4716.40 \\
	Elastic Net & 0.37 & 6514.98 \\
	Decision Tree & 0.04 & 8042.44 \\
	Random Forest & 0.35 & 6628.65
\end{tabular}

%%%% ORDINARY LEAST SQUARES %%%%
% negative residuals
\begin{tabular}{c c c}
	\textbf{Rank} & \textbf{Institution Name} & \textbf{Residual} \\ 
	\hline
	1 & Rutgers University-Newark & -12387.28\\
	2 & Pennsylvania State University-Penn State Lehigh & -11728.43 \\ 
	3 & Pennsylvania State University-Penn State York &  -11050.88\\
	4 & Rutgers University-Camden & -9411.65 \\
	5 & Swedish Institute a College of Health Sciences & -8643.17\\
	6 & Florida National University-Main Campus & -8355.56 \\
	7 & Baker College of Flint & -8024.05 \\
	8 & SUNY at Purchase College & -7550.39 \\
	9 & South University-The Art Institute of Dallas & -7488.18 \\
	10 & Oklahoma City University & -7364.77 
\end{tabular}
% positive residuals
\begin{tabular}{c c c}
\textbf{Rank} & \textbf{Institution Name} & \textbf{Residual} \\ 
	\hline
	1 & Westwood College-Los Angeles & 12259.81 \\
	2 & Fairfield University & 11905.85 \\ 
	3 & Savannah State University &  11050.12\\
	4 & Prescott College & 8524.98 \\
	5 & Auburn University & 6719.69\\
	6 & Pacific Oaks College & 6449.84 \\
	7 & Rose-Hulman Institute of Technology & 6331.67 \\
	8 & Silver Lake College of the Holy Family & 5702.42 \\
	9 & Daemen College & 4529.33 \\
	10 & University of Colorado Boulder & 4446.70
\end{tabular}

%%%%DECISION TREES%%%%
% negative residuals
\begin{tabular}{c c c}
\textbf{Rank} & \textbf{Institution Name} & \textbf{Residual} \\ 
	\hline
	1 & Eastern Connecticut State University & -30900.00 \\
	2 & Southern University and A \& M College & -26800.00 \\ 
	3 & Auburn University & -26000.00 \\
	4 & Savannah State University & -23600.00 \\
	5 & Touro College & -20116.67 \\
	6 & Hendrix College & -18692.31 \\
	7 & California College of the Arts & -15392.31 \\
	8 & Northwestern University & -15100.00 \\
	9 & Valley Forge Christian College & -13338.78 \\
	10 & Naropa University & -12325.00
\end{tabular}

% positive residuals
\begin{tabular}{c c c}
\textbf{Rank} & \textbf{Institution Name} & \textbf{Residual} \\ 
	\hline
	1 & Rose-Hulman Institute of Technology & 31900.00 \\
	2 & Butler University & 21200.00 \\ 
	3 & Bryan College of Health Sciences & 18541.90 \\
	4 & University of Oklahoma-Health Sciences Center & 17803.13 \\
	5 & Washington Adventist University & 10862.69 \\
	6 & Pace University-New York & 10403.13 \\
	7 & CUNY Bernard M Baruch College & 9521.54 \\
	8 & The University of Texas at Dallas & 8821.54 \\
	9 & Silver Lake College of the Holy Family & 8820.00 \\
	10 & Stony Brook University & 8721.54
\end{tabular}

%%%%KNN%%%%
% negative residuals
\begin{tabular}{c c c}
\textbf{Rank} & \textbf{Institution Name} & \textbf{Residual} \\ 
	\hline
	1 & Naropa University & -20060 \\
	2 & Florida National University-Main Campus & -16760 \\ 
	3 & California College of the Arts & -16720 \\
	4 & Metropolitan State University of Denver & -15960 \\
	5 & Eastern Connecticut State University & -14940 \\
	6 & Hendrix College & -14400 \\
	7 & Prescott College &  -13000 \\
	8 & Thomas University & -10320 \\
	9 & Westmont College & -9480 \\
	10 & New York College of Health Professions & -9300
\end{tabular}

% positive residuals
\begin{tabular}{c c c}
\textbf{Rank} & \textbf{Institution Name} & \textbf{Residual} \\ 
	\hline
	1 & Rose-Hulman Institute of Technology & 24200 \\
	2 & University of Oklahoma-Health Sciences Center & 21940 \\ 
	3 & New York School of Interior Design & 15020 \\
	4 & Bryan College of Health Sciences & 14800 \\
	5 & Butler University & 11240 \\
	6 & CUNY Bernard M Baruch College & 11240 \\
	7 & Michigan Technological University & 10020 \\
	8 & Pace University-New York & 8340 \\
	9 & The University of Texas at Dallas & 8220 \\
	10 & Daemen College & 7140
\end{tabular}

\end{center}

 
\end{homeworkProblem}
\pagebreak

%\pagebreak

%
% Non sequential homework problems
%

% Jump to problem 18
%\begin{homeworkProblem}[18]
%    Evaluate \(\sum_{k=1}^{5} k^2\) and \(\sum_{k=1}^{5} (k - 1)^2\).
%\end{homeworkProblem}
%
%% Continue counting to 19
%\begin{homeworkProblem}
%    Find the derivative of \(f(x) = x^4 + 3x^2 - 2\)
%\end{homeworkProblem}
%
%% Go back to where we left off
%\begin{homeworkProblem}[6]
%    Evaluate the integrals
%    \(\int_0^1 (1 - x^2) \dx\)
%    and
%    \(\int_1^{\infty} \frac{1}{x^2} \dx\).
%\end{homeworkProblem}

\end{document}